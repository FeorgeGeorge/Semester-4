\section{Введение}

Теория вероятности изучает случайные события, а эксперименты, в ходе которых они случаются, называется нами случайным экспериментом или часто испытанием.

\section{Статистическое определение вероятности}
$n$ экспериментов, $n_a$ --- число появления событий $A$, $\frac{n_a}{n}$ -- относительная частота события $A$.
\begin{definition}
    $P(A) = \frac{n_a}{n}$ --- вероятность события $A$ при большом $n$.
\end{definition}

Это интуитивное определение, но эта интуиция может быть обманчива.
У этого определения есть ряд недостатков (эксперименты часто провести нельзя, в зависимости от числа экспериментов относительная частота разнится и т.д) и мы не будем им пользоваться.

В основе хорошего определения вероятности лежит концепция вероятностного пространства, в котором случайное событие понимается как подмножество. 
\subsection{Пространство элементарных исходов}
\begin{definition}
    Пространством элементарных исходов $\Omega$ назывыается множество, содержащее все возможные исходы случайного эксперимента, из которых при испытании происходит только один.
\end{definition}

\begin{definition}
    Случайными событиями называются подмножества $A\subset \Omega$.
    В ходе эксперимента событие $A$ наступило, если произошел один из элементарных исходов, входящиъ в множество $A$. 
\end{definition}

Не все подмножества пространства исходов понимаются нами как события.
Некоторые события могут быть нам попросту неинтересны.
Некоторые мы не можем считать событиями. 
Например, в дискретном случае событие с одним исходом считается событием, но в непрерывном, как правило, нет.
<Примеры>

\subsection{Операции над событиями}
$\Omega$ --- достоверное событие, или универсальное событие.
$\emptyset$ --- невозможное событие, или пустое. 
Никогда не происходит, потому что не содержит исходов.

\begin{definition}
    Суммой $A+B$ называется событие $A \cup B$, сосоящее в том, что произошло событие $A$ или событие $B$ (хотя бы одно из них).
\end{definition}

\begin{definition}
    Произведением $AB$ называется  событие $A\cap B$, сосоящее в том, что произошло событие $A$ и событие $B$ (т.е оба из них, одновременно, в ходе одного эксперимента). 
\end{definition}

$A_1 + \ldots A_n$ --- произошло хотя бы одно.
$B_1 \cdot \ldots \cdot B_n$ --- произошли все. 
Do you get it yet?

\begin{definition}
    Противоположным к $A$ называется событие $\overline{A} = \Omega \setminus A$, состоящее в том, что событие не произошло.
\end{definition}

Заметка: $\overline{\overline{A}} = A$.


\begin{definition}
    Разницей событий $A$ и $B$ называется событие $A-B = A \overline{B}$.
\end{definition}

\begin{definition}
    События $A$, $B$ называются несовместными если они не пересекаются, $A\cap B= \emptyset$, т.е одно исключает появление другого в эксперименте.
\end{definition}

\begin{definition}
    Событие $A$ влечёт событие $B$, если $A\subset B$. 
    В логике это соотвествует импликации: если появилось $A$, то появилось $B$. 
\end{definition}


\subsection{Вероятность}
Вероятность события $A$~--- некая числовая характеристика $P(A)$, $0 \leqslant P(A) \leqslant 1$, отражающая частоту наступления $A$ в эксперименте.

Пусть $\Omega$ содержит конечное число равновозможных исходов (условно, из соображений симметрии).
Тогда применимо следующее классическое определение
\begin{definition} 
    $P(A) = \frac{|A|}{|\Omega|}$~--- вероятность $A$.
\end{definition}

Свойства вероятности. 
\begin{enumerate}
    \item $0 \leqslant P(A) \leqslant 1$
    \item $P(\Omega) = 1$
    \item $P(\emptyset) = 0$
    \item Если $A$ и $B$ несовместны, то $P(A + B) = P(A) + P(B)$ (доказательство есть--- оно простое)
\end{enumerate}

Это определение применимо в очень ограниченном числе случаев.

\section{Геометрическое определение вероятности}
Пусть $\Omega$ --- это замкнутая ограниченная область в $\mathbb{R}^n$.
Пусть $\mu$ --- мера на $\Omega$.

Мы ''наугад'' бросаем точки в эту $\Omega$. 
Наугад значит, что вероятность попадание в область зависит только от её меры (длины, площади, обьема и т.д).
В этом случае применимо геометрическое определение вероятности.

\begin{definition}
    $P(A) = \frac{\mu(A)}{\mu(Omega)}$.
\end{definition}
\begin{remark}
Свойства этой вероятности повторяют свойства классических.
\end{remark}

<Примеры>
 
Геометрическая вероятность упускает случай счётного множества элементарных исходов.
Она прменяется крайне-крайне редко.

На следующей лекции будет нормальное формальное точное определение вероятности.

Административные вопросы: телеграм-канал потока $tv3234_37$.
3 online контрольных без переписываний (переписывания в крайнем случае с штрафом).
Получить 4 до экзамена можно за контрольные. 
5 требует экзамен.

Для очников: 10 за посещение, 10 за работу на парах, 60 за контрольные.





 