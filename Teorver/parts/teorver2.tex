\section{Сигма алгебра}

\begin{definition}
Пусть $\Omega$ --- пространство элементарных исходов, $F$~--- $\sigma$-алгебра на $\Omega$.
Вероятностью на $\Omega$ называется функция
\begin{itemize}
    \item $P(A) \geqslant 0$
    \item Вероятность (счётной) суммы несовместных событий --- сумма вероятностей.
    \item Нормированность. $P(\Omega) = 1$. 
\end{itemize}
\end{definition}

\begin{definition}
Вероятностынм пространством называется совокупность пространства элементарных исходов, сигмы алгебра на нём и вероятности.
\end{definition}

Формула обратной вероятности $P(A) = 1 - P(\neg A).$

Аксиома непрерывности. 
При непрерывном изменении области соотвествующая вероятность также должна изменяться непрерывно.

Аксиома непрерывности следует из аксиомы счётной аддитивности.

Независимые события. 
Независимость набора событий.
Пример Бернштейна.


\endinput