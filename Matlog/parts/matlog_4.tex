\section{Алгебра Гейтинга}

\begin{definition}
    Алгебра Гейтинга --- это импилкативная решетка с 0.
\end{definition}

\begin{definition}
    $\neg a \equiv a\to 0$.
\end{definition}

\begin{definition}
    Булева алгебра --- алгебра Гейтинга, где $a + \neg a = 1$.
\end{definition}

Пример. 

\begin{definition}
    $\alpha \preccurlyeq \beta$, если $\alpha \vdash \beta$.
\end{definition}

\begin{definition}
    $\alpha \approx \beta$, если $\alpha \preccurlyeq \beta$ и $\beta \preccurlyeq \alpha$.
\end{definition}
 
\begin{definition}
    Пусть $E$~--- множество всех высказываний ИИВ.  
    Тогда $[E]_{\approx}$ --- фактор-множество высказываний по отношению $\approx$, 
    называется алгеброй Линденбаума, $L$.
\end{definition}

\begin{theorem}
    Алгебра Линденбаума --- это алгебра Гейтинга.
\end{theorem}

\begin{theorem}
    $L$ --- полная и корректная модель ИИВ.
\end{theorem}

\begin{theorem}
    Алгебра Гейтинга --- полная и корректная модель ИИВ.
\end{theorem}

\begin{definition}
    Исчисление дизьюнктно, если для любых $\alpha, \beta$ доказуемость $\alpha \lor \beta$ влечет доказуемость или $\alpha$ или $\beta$.
\end{definition}

\begin{theorem}
    ИИВ дизьюнктно.
\end{theorem}

\begin{definition}
    Пусть $A, B$ --- алгебры Гейтинга. 
    $f$ называется гомоморфизмом, если $f(0_A) = 0_B$, $f(1_A) = 1_B$, $f (\alpha \star_A \beta) = f(\alpha) \star_B f(\beta)$.
\end{definition}

\begin{definition}
    Гёделева алгебра --- та, в которой $a + b = 1$ влечёт $a=1$ или $b=1$.
\end{definition}

\begin{definition}
    Пусть $A$ --- алгебра Гейтинга.
    Определим $\Gamma (A)$.
    $\gamma(x) = \begin{cases}
        \omega, & x = 1_A \\
        x, & x < 1_A \\
    \end{cases}$.
    Добавим $1_B$, которое больше каждого $\gamma(t)$. 
\end{definition}

\begin{theorem}
    $\Gamma (A)$ --- гёделева алгебра.
    В этой алгебре $\Gamma (A)$ выполнено $a + b = 1$ влечёт $a=1$ или $b=1$.  
\end{theorem}

\begin{theorem}
    $\Gamma (L)$ гёделева. 
\end{theorem}

Определим $\gamma : \Gamma(L) \to L: \quad g(x) = \begin{cases}
    1, & x = 1\ \text{или}\ x = \omega \\
    x, & \text{иначе} 
\end{cases}$. 
Утверждается, что это гомоморфизм.

Рассмотрим ИИВ и алгебры $L, \Gamma (L)$.
Рассмотрим $\vdash \alpha \lor \beta$.
$\Gamma (L)$ --- геделева алгебра, т.е алгебра Гейтинга.

$\ldots $
% \[\llbracket \alpha \lor \beta \rrbracket \]

\begin{definition}
    Модель ИИВ называется табличной, если множество истиностных значений $V = S$, $\llbracket \alpha \star \beta \rrbracket = f_\star (\llbracket \alpha \rrbracket, \llbracket \beta \rrbracket)$.
    и есть два значения, выделенная истинна И, и $\llbracket \alpha \rrbracket = \text{И}$ в том и только том случае, когда $\vDash \alpha$.     
\end{definition}

\endinput