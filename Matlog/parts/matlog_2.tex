\section{Теорема о дедукции}

\begin{definition}
    (Метаметаопределение).
    Будем большими греческими буквами $\Gamma, \Delta, \Sigma\ldots$ --- списки формул, неупорядоченные. 
\end{definition}

\begin{definition}
    Вывод из гипотез: $\Gamma \vdash \alpha$ (см. лекцию 1) 
\end{definition}

\begin{theorem}
    $ \Gamma, \alpha \vdash \beta$ тогда и только тогда, когда $\Gamma \vdash \alpha \to \gamma$. 
\end{theorem}

\begin{proof}
    \begin{itemize}
        \item [$\Leftarrow $] 
        Пусть $\delta_1, \delta_2\ldots \delta_n \equiv \alpha \to \beta$ выводит $\alpha \to \beta$. 
        Дополним этот вывод двумя новыми высказываниями: $\delta_{n+1} \equiv \alpha$ (дано нам в гипотезе), $\gamma_{n+2} \equiv \beta$ (MP шагов $n$, $n+1$) --- это и требовалось.
        \item Напишем программу, которая трансформирует один вывод в другой.
        Инвариант, который мы будем поддерживать: всё до $\alpha \to \delta_i $ --- док-во.
        Доказательство индукцией по $n$.
        \begin{enumerate}
            \item База: $n=1$ --- без комментариев.
            \item Если $\delta_1,\ldots, \gamma_n$ можно перестроить в доказательство $\alpha \to \gamma_n$, то $\gamma_1 \ldots \gamma_{n+1}$ тоже можно перестроить.
            Разберём случаи:
            \begin{itemize}
                \item $\delta_i$ --- гипотеза из $\Gamma$. Тогда 
                \begin{itemize}
                    \item[$(i-0.1)$] $\delta_i$ (аксиома или гипотеза)
                    \item[$(i-0.2)$] $\delta_i \to \alpha \to \delta_i$ (схема 1)
                    \item[$(i)$] $\alpha \to \delta_i$ (MP )
                \end{itemize}
            \end{itemize}
            ДОПОЛНИТЬ
        \end{enumerate}
    \end{itemize}
\end{proof}

\section{Теория моделей}
Мы можем докаывать модели или оценивать их.
''Мы можем доказать, что мост не развалится или можем выйти и попрыгать на нём.''

\begin{definition}
    $\mathbb{V}$~--- истинностное множество.

    $F$~--- множество высказываний нашего исчисления высказываний.

    $P$~--- множество пропозициональных переменных.
    \[ \llbracket\cdot\rrbracket: F \to \mathbb{V} \text{~--- оценка}\]
\end{definition}

\begin{definition}
    Для задания оценки необходимо задать оценку пропозициональных переменных.
    \[\llbracket \cdot \rrbracket : P\to \mathbb{V} \quad f_P\]
    Тогда: 
    \[ \llbracket x \rrbracket = f_p(x)\]
\end{definition}

\begin{remark}
    Обозначение: значения пропозициональных переменных будем определять в верхнем индексе: $\llbracket \alpha \rrbracket ^{A = T, B = F \ldots}$
\end{remark}

\begin{definition}
    $\alpha$ --- общезначна (истинна), если $\llbracket \alpha \rrbracket = T$ при любой оценке $P$.

    $\alpha$ --- невыполнима (ложна), если $\llbracket \alpha \rrbracket = F$ при любой оценке $P$.
    
    $\alpha$ --- выполнима, если $\llbracket \alpha \rrbracket = T$ при некоторой $f_P$.

    $\alpha$ --- опровержима, если $\llbracket \alpha \rrbracket = F$ при некоторой $f_P$.
\end{definition}

\begin{definition}
    Теория корректна, если доказуемость влечёт общезначимость.

    Теория полна, если общезначимость влечёт доказуемость.
\end{definition}

\begin{definition}
    $\Gamma \vDash \alpha$, $\alpha$ следует из $\Gamma = \{ \gamma_1, \ldots, \gamma_n\}$, 
    если $\llbracket \alpha \rrbracket = T$ всегда при $\llbracket \gamma_i \rrbracket = T$ для любых $i$.  
\end{definition}

\begin{theorem}
    Исчисление высказываний корректно, т.е
    \[ \vdash \alpha\ \text{влечёт}\ \vDash \alpha. \]
\end{theorem}
Мы даём доказательство на метаязыке, не пускаясь в отчаянный формализм.
Такая строгость нас устраевает.
\begin{proof}
    Индукция по длине доказательства. 
    Не очень сложно. 
    ДОПОЛНИТЬ.
\end{proof}

В матлогике бесмысленно формализовывать русский язык.
Она нужна, чтобы дать ответы на сложные вопросы в математике, где здравого смысла недостаточно и нужна формализация.

\section{Полнота исчисления высказываний}
\begin{theorem}
    Исчисление высказываний полно.
\end{theorem}

\begin{definition}
    $_{[\beta]}\alpha = 
    \begin{cases}
        \alpha, & \llbracket \beta \rrbracket = T\\
        \neg \alpha,&  \llbracket \beta \rrbracket = F\\
    \end{cases}$
\end{definition}

\begin{lemma}
    $_{[\alpha]}\alpha,\ _{[\beta]}\beta \vdash _{[\alpha \star \beta]} \alpha \star \beta$ 

    $_{[\alpha]} \alpha \vdash _{[\neg \alpha]}\neg\alpha $
\end{lemma}

\begin{lemma}
Если $\Gamma \vdash \alpha$, то $\Gamma, \Delta \vdash \alpha$.
\end{lemma}

\begin{lemma} 
Пусть дана $\alpha$, $X_1,\ldots, X_n$ --- её переменные.   

\[ _{[X_1]}X_1,\ldots _{[X_n]}X_n\ \vdash _{[\alpha]}\alpha  \]
\end{lemma}
\begin{proof}
    Индукция по структуре. ДОПОЛНИТЬ
\end{proof}

Сократим запись и вместо этой кучи $X$ будем писать $X^\prime$.
\begin{lemma}
    Если $\vDash \alpha$, то $X^\prime \vdash \alpha$.
\end{lemma}

\begin{lemma}
    \[ \Gamma, Y \vdash \alpha, \quad \Gamma, \neg Y \vdash, \text{ то } \Gamma \vdash \alpha \]
\end{lemma}

\begin{theorem}
    Если $\vDash \alpha$, то $\vdash \alpha$.
\end{theorem}

\section{Интуиционистская логика}
Мы не хотим дурацких коснтрукций вроде парадокса брадобрея.
Мы не хотим странных, но логически верных утверждений вроде $A\to B \lor B \to A$.
Интуиционисткая логика предлагает свою математику, в которой своя интерпретация логических связок.
ВНК-интерпретация (Брауер-Гейтинг-Колмогоров).

\begin{itemize}\itemsep=-1mm
    \item $\alpha, \beta, \gamma \ldots$ --- это конструкции.
    \item $\alpha \land \beta$ если мы умеем строить и $\alpha$, и $\beta$.
    \item $\alpha \lor \beta$, если мы умеем строить $\alpha, \beta$ и знаем, что именно.
    \item $\alpha \to \beta$, если мы умеем перестроить $\alpha$ в $\beta$.
    \item $\bot $ --- не имеет построения
    \item $\neg \alpha \equiv \alpha \to \bot$   
\end{itemize}

''Теория доказательств''.
Рассмотрим классическое исчисление высказываний и заменим схему аксиом 10 на следующую 
\[ \alpha \to \neg \alpha \to \beta \] 

В этой формализации мы следуем не сути интуиционисткой логики, а традиции.
В интуиционисткой логике формализм это не источник логики.

Примеры моделей.
\begin{enumerate}
    \item Модели КИВ подходят: корректны, но не полны.
    \item Пусть $X$ топологическое пространство.
\end{enumerate}


\endinput