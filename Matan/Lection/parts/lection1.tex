\section{Повторение}
Задачи и темы, которые мы будем обсуждать в новом семестре: 
многообразия, дифференциальные формы на них, криволинейные интегралы, интегралы от параметров, формула Стокса, формула Остроградского, $\gamma$-, $\beta$-функции.

Интеграл от функции произвольного знака это разность интегралов компонент. 
% \[ \]
В случаях, когда оба слогаемых не бесконечные, такая разность имеет смысл.

Интеграл комплекснозначной функции это сумма интегралов вещественных компонент функции.   
\[ 
    \int_E f d\mu = \int_E \Re f d\mu + \int_E \Im f d \mu 
\]


Монотонность интеграла.
\[ \int_E(f_1 + f_2) \geqslant \int_E f_1 = \infty  \]

\begin{theorem}[Теорема Леви для последовательности]
Если $f_n$ неотрицательные измеримые на $E$ функции и $f_n \uparrow f$ возрастая сходится поточечно к $f$, то 
\[ 
    \lim \int_E f_n d\mu = \int \lim f_n d\mu = \int f d\mu 
\]
\end{theorem}

\begin{theorem}[Теорема Леви для рядов]
Если $f_n$ неотрицательные измеримые на $E$ функции, то интеграл от ряда совпадает с суммой ряда из интегралов.
\[ \int_E \sum_{n = 1}^\infty f_n d\mu = \sum_{n=1}^\infty = \sum_{n=1}^{\infty} \int_E f_n d\mu\]
\end{theorem}

\begin{proof}
    Пусть $S_n(x) = \sum_{n=1}^{\infty} f_n(x)$~--- частичная сумма.
    $S(x) = \sum_{n=1}^\infty f_n(x) = \lim_{n\to \infty} S_n(x)$ 
\end{proof}

\begin{example}
Функция, которая не удовлетворяет условиям теоремы Леви: 
\[ f_k(x) = \xi_{[k,k+1]}(x) \]

\[ \int_{[0, +\infty]} f_k(x)d\mu = \int_{[k,k+1]} f_k(x)d\mu = 1\]
\[ \int f(x) d\mu = \int_{[0, +\infty]} 0 d\mu = 0\]
\end{example}

\begin{remark}
\begin{enumerate}
\item Для $f\in S(E)$  $|f| \in L(E, \mu)$ тогда и только тогда, когда $f\in L(E, \mu)$.
\item Если интеграл $\int_E f d\mu$ определен, то $\int_E |f|d \mu \geqslant |\int_E f d\mu|$.  
\end{enumerate}
\end{remark}

\begin{proof}

\end{proof}
Отсутпление про суммируемую мажоранту.

Если функция имеет суммируемую мажоранту, то сама она является суммируемой.

...
$L_1(E, \mu)$:
две функиции эквивалентны по мере на $E$, если они совпадают почти везде на $E$.
Другими словами, мера подмножества $E$, на котором функции принимают разные значения, равна нулю.
\[ \| f \|_1 = \int_E |f|d\mu \].

Элементы $L_1(E, \mu)$ могут быть определены не на всём $E$ целиком, но на множестве полной меры.
\[ |f+g| \leqslant |f| + |g| \]

Эта норма невырожденная. 
Если $f\in S_{+}(E)$ и $\int f \mu = 0$, то $f = 0$ почти всюду на $E$. 

\begin{theorem}
    [Счётная аддитивность интеграла]
Пусть $f\in S(E)$ $E = \bigcup_{k=1}^\infty E_k$, 
$E_k \in ?$, определн $\int_E f d\mu$. 
Тогда 
\[ 
\int_E f d\mu = \sum_{k=1}^\infty \int_{E_k} f d\mu
\]
\end{theorem}

\begin{proof}
...
\end{proof}

\begin{theorem}
[О приближении интеграла интегралом по множеству конечной меры]

Пусть мера $E$ конечна и $f\in L(E, \mu)$ суммиурема.
Тогда 
\[
    \forall \epsilon> 0 \exists E_0 \subset E: \mu(E_0) < +\infty \text{и} \int_{E\setminus E_0} |f| d\mu < \epsilon
\]  
\end{theorem}

\begin{proof}
Не умаляя общности $f \geqslant 0$ на $E$. 
Продложим $f$ нулем вне $E$. $J(A) = \int_A f d\mu$--- мера.
$E_K = E\{ f > \frac{1}{k}\}$, $E_* = E\{f > 0\} = \bigcup_{k=1}^\infty E_k$.

Непрерывность меры снизу
$E_k$ --- множества конечной меры.

Научились приближать с любой точностью интеграл интегралом по множествам конечной меры.
\end{proof}

Теорема Фато и теорема Лебега.
\begin{theorem}
    Пусть $f_k\\in S_+(E)$ для всех $k\in \mathbb{N}$.
    Тогда $\underline{\lim}_{k\in \infty} \leqslant \underline{\lim}\int_E f_k(x)$. 

    И если $f_k(x) \to f(x)$ на $E$, то $\int_E f(x) \leqslant \underline{\lim} \int_E f_k(x)$
\end{theorem}

\begin{theorem}
[Теорема Лебега о мажорированной сходимости]
Пусть $f_n \to f$ сходится почти везде на $E$ и $\Phi \in L(E, \mu)$:
$\forall k\in \mathbb{N} |f_k| \leqslant \Phi$ почти везде на $E$.
Тогда $f\in L(E, \mu)$ и $\lim_{k\to\infty} \int_E fd\mu$.
...
\end{theorem}

Интеграл положительнозначной функции определяет меру. 
Интеграл функции это разность мер компонент. 
Такая разность называется заряд. 
