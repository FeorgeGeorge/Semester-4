\section{Введение}
Базовые технологии: ввод-вывод, collections framework, потоки, рефлексия, разные инструменты.
Потом большой блок многопоточного программирования.
На базе этог блока мы разберёмся с сетевыми приложениями.
Затем приступим к блоку ''разное'', принимаются адекватные предложения по поводу доп. лекций.

\section{Ввод-вывод}
Потоки различаются по направлению: ввод или вывод.
И по содержимому: байтовые и символьные.
У Java двухбайтные символы, этим различается символьный от байтового.

\begin{tabular}{ccc}
    & Байтовый & Символьный \\
    Ввод & InputStream & Reader \\
    Вывод & Outputstream & Writer \\
\end{tabular}

\section{Исключительные ситуации}
IOException --- это корень иерархии исключений ввода-вывода.

EOFEception --- попытка чтения после конца потока.

FileNotFoundException --- файл не найден.

UnsupportedEncodingException --- неизвестная кодировка
Есть кодировки, которые не поддерживаются некоторыми JDK.
Это плохой дизайн. 
Во-первых, как это обрабатывать --- это вопрос.
Во-вторых, проблеме программа узнает, когда она пытается открыть файл в данной кодировке, например, но этих мест в программах может быть очень много.

Проверяемые нужно обрабатывать, не нужно обрабатывать непроверяемые.

\section{Потоки ввода}
В сигнатурах тип T означает байт для байтовых потоков и символьный ип для символьных потоков соотвественно.

\def\cmd#1 ---{\ttfamily #1\normalfont~---}

\begin{itemize}\itemsep=-1mm
    \item \cmd int read() --- чтение элемента.
    \item \cmd read(T[] array), read(T[] array, offset, length) --- чтение элементов в массив.
    \item \cmd byte[] readNBytes(n) --- чтение массива.
\end{itemize}
Метод read не гарантирует, что он прочтёт ровно столько данных, сколько вы попросили.
Файл может кончиться. 
Тогда read даст вам хвост и вернет -1.

Некоторые потоки поддерживают дополнительные операции:
\begin{itemize}\itemsep=-1mm
    \item \cmd skip(n) --- пропуск до $n$ элементов.
    \item \cmd avaliable() --- сколько элементов доступно.
\end{itemize}

Закрытие потока.

Интерфейс Closeable extends 
$\ldots$

\section{Пометки и возвраты}
\cmd markSupported() --- проверка поддержки.

Основные операции:
\begin{itemize}\itemsep=-1mm
    \item \cmd write(v) --- запись элемента.
    \item \cmd write(T[] array) --- запись массива элементов
    \item \cmd write(T[] array, offset, length) --- запись фрагмента.
\end{itemize}

Запись буфера:
\begin{itemize}\itemsep=-1mm
    \item Интерфейс Flushable
    \item Метод flush
\end{itemize}

Успешное закрытие потока означает, что все данные были записаны.

\section{Файловый ввод-вывод}
Кодировка символьных потоков по умолчанию зависит от системы.
Почти всегда неуказание конкретной кодировки будет ошибкой.

<Обсуждение кода>
\section{Фильтрующие потоки}



\endinput
