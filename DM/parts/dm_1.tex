Последний семестр дискретной математики. 
Две больших темы: производящие функции (комбинаторика) и введение в теорию вычислимости. 

\section{Производящие фукнции}

Рассмотрим последовательности $\{ a_n\}_{n\in \mathbb{N}}, \{b_n\}_{n\in \mathbb{n}} \subset \mathbb{R}(\mathbb{C})$.
Назовём эти последовательности $A$ и $B$ и будем почленную сумму обозначать кратко $A+B$.
Это несколько неудобно и неестественно, об этих конвенциях нужно договариваться.

Вместо этого давайте рассмотрим формальный степенной ряд, у которого члены последовательности это коэффициенты ряда.
\[ A(t) = a_0 + a_1 t + a_2 t^2 + \ldots a_n t^n + \ldots\]
Тогда почленная сумма последовательностей будет соотвествовать обычной сумме рядов $A(t) + B(t)$.

Чтобы сдвинуть последовательность на 1 вправо, можно просто умножить степенной ряд на $x$. 

Можем рассмотреть степенной ряд-композицию $A(t^2) = a_0 + 0 t + a_1 t^2 + \ldots$. 
Это степенной ряд, соответвующий последовательности $a_0, 0, a_1, 0, a_2 \ldots$.

Таким образом, мы можем ''оперировать'' над последовательностью как единым целым, и это очень удобно.

Мы не рассматриваем степенные ряды с стороны, с которой на них смотрит мат. анализ: как способ приблизить фукнцию, с некоторым радиусом сходимости и т.д.
У нас степенные ряды {\it формальные} и не всегда (всегда не) должны пониматься как функции, в которой в переменную можно подставить значение.

$\mathbb{R}[x]$ --- кольцо многочленов с коэффициентами из кольца $R$, состоящий из формальных многочленов.
$\mathbb{R}[x]^+$~--- множество формальных степенных рядов.

\begin{definition}
Формальный степенной ряд $A(t)$ последовательности $\{a_n\}_{n\in \mathbb{N}}$ называется производящей функцией (generating function).  
\end{definition}

Название неудачное. 
Оно связано с другими корнями понятия производящей функции (они нужны не только в комбинаторике).

Определена сумма производящих функций и произведение 
\[ A(t) B(t) = C(t)\quad c_n = \sum_{i=0}^n a_ib_{n-i} \]

Несмотря на то, что мы работаем с бесконечным по размеру обьектом, нам необходимо только конечное число элементов, чтобы посчитать каждый отдельынй его член.
Этот раздел дискретной математики не любит предельных переходов.

Определено умножение на скаляр.
\[ \lambda A(t) = C(t) \quad c_n = \lambda a_n\]

Определено даже деление! 
\[ \frac{A(t)}{B(t)} = C(t);\ b_0 \neq 0 \quad c_n = \frac{a_n - \sum_{i=0}^{n-1}c_ib_{n-i}}{b_0} \] 
Так можно посчитать, например, что 
\[C(t) = \frac{1}{1-t} = 1 + t + t^2 + \ldots t^n + \ldots; \quad a_n = 1\]
Мы записали короткой (конечной) производящей функцией бесконечную последовательность. 
Более того, мы можем эту запись взять и производить с ней операции (умножать и складывать с другими производящими функциями).

\[ \frac{1}{1-2t} = 1 + 2t + 4t^2 + \ldots + 2^n t^n; \quad c_n = 2^n \]

Обобщая мы видим, что 
 \[ \frac{1}{1-bt} = \sum_{n=0}^{\infty}b^n t^n = C(bt)\]
 Вообще говоря,
 \[ A(t) = \sum a_nt^n \quad A(bt) = \sum a_n b^n t^n\]

 \begin{remark}
 Если $b_0 = \pm 1$, $a_i, b_i \in \mathbb{Z}$, тогда $C = \frac{A}{B}$ с целочисленными коэффициентами $c_i \in \mathbb{Z}$.
 \end{remark}    

\[ \frac{1}{1-t-t^2} = 1 + t + 2t^2 + \ldots + F_nt^n + \ldots\]
Мы одной дробью породили целую последовательность Фиббоначи!

А как быть, если мы хотим взять последовательность и найти представление для её производящей функции?
Мы можем поступить так.
\[ F_0 = 1,\ F_1 = 1,\ F_n = F_{n-1} + F_{n-2}\]
Отсюда $F(t) + F(t)t = \frac{F(t)-1}{t}$ (следует из операций над производящими функциями и рекурентным соотношением последовательности фиббоначи).

А что с дифференцированием? 
Обыкновенная операция взятия производной (формального) степенного ряда позволяет нам умножать член последовательности на его номер.
\[ A(t) \to A(t)^\prime \cdot t \] 
Эту операцию можно производить многократно, получая последовательноть членов исходной п-ти в $k$ степени.

Как найти представление производящей функции для последовательности $a_n = n$?
\[ a_n = n * 1\]
Производящая функция для п-ти единиц это $\frac{1}{1-t}$.
Тогда 
\[ A(t) = \left( \frac{1}{1-t} \right)^\prime t = \frac{t}{(1-t^2)} \]
Формальное деление это подтверждает.

А что с интегрированием? 
С интегрированием всё не очень мило.

А что с композицией? 
\[C(t) =  A(B(t))\]
Здесь много проблем доставляет свободый коэффициент у $B$.
Давайте его уберем --- $b_0 = 0$.
Теперь мы можем посчитать 
\[ c_n = \sum_{k=0}^{n}a_k \sum_{n = i_1 + i_2 + \ldots i_k}b_{i_1}b_{i_2}\ldots b_{i_k} \]

Пример с доминошками.

Пример с деревьями.
